% Created 2022-09-21 Wed 15:57
% Intended LaTeX compiler: pdflatex
\documentclass[11pt]{article}
\usepackage[utf8]{inputenc}
\usepackage[T1]{fontenc}
\usepackage{graphicx}
\usepackage{longtable}
\usepackage{wrapfig}
\usepackage{rotating}
\usepackage[normalem]{ulem}
\usepackage{amsmath}
\usepackage{amssymb}
\usepackage{capt-of}
\usepackage{hyperref}
\author{brickviking}
\date{\textit{<2022-09-21 Wed>}}
\title{fossil scripts\\\medskip
\large for source code maintenance and forum viewing on localhost}
\hypersetup{
 pdfauthor={brickviking},
 pdftitle={fossil scripts},
 pdfkeywords={},
 pdfsubject={},
 pdfcreator={Emacs 29.0.50 (Org mode 9.5.4)}, 
 pdflang={English}}
\begin{document}

\maketitle


\section*{Introduction}
\label{sec:org3d15dbd}
This is a set of homebaked scripts to fetch fossil-related source code such as the fossil program
or sqlite. It's currently implemented as bash code, with no real intention to rework it to anything
else. Hey, if it works, \ldots{} right?

\section*{Notes}
\label{sec:org7b2d92d}
We should probably make sure the fossil directory and the sqlite directories exist before we
try to tangle code into them.

\section*{Changelog}
\label{sec:org8c0e5af}
\begin{itemize}
\item 1.0 Initial fossil.sh created, as version 0.0.1, pretty much just fetched fossil-scm source
code.
\begin{itemize}
\item 1.1 expanded to fossilsource.sh, includes forums, added fossil book a while later.
\item 1.2 included fossilweb.sh, initially starts up servers to access fossil source and forums
\item 1.3-1.4 changes
\item 1.5 added sqlite source code fetching and forums to fossilsource.sh
\item 1.6 added sqlite web server to fossilweb.sh
\item 1.7 split off sqliteweb stuff into its own script (sqliteweb.sh)
\end{itemize}
\item 2.0 expanded fossilsource to create functions for fetching individual components.
\begin{itemize}
\item 2.1 added docsrc (sqlite) to fossilsource.sh and sqliteweb.sh
\item 2.2 added sqllogictests to fossilsource.sh and sqliteweb.sh

\begin{itemize}
\item 2.3 Initial cut of a fossil.org containing document, now that these scripts are getting big
enough.
\end{itemize}
\end{itemize}
\end{itemize}
\newpage

\setcounter{tocdepth}{2}
\tableofcontents

\newpage

\section*{Source files}
\label{sec:org57d56bf}
These three files are responsible for updating source fossil files, or starting up the internal
content web servers related to fossil or sqlite.

\subsection*{fossilsource.sh}
\label{sec:org9b5a113}
This fetches all the code that's currently supplied as .fossil files from sqlite.org, including
the sqlite-related files.
As of September 2022, that's the following files stored in \$\{HOME\}/src/c/fossil-scm:
\begin{itemize}
\item fossil-scm source (fossil.fossil)
\item fossil-scm fossil (fossilforum.fossil)
\item fossil-scm book (fossil-book.fossil)
\end{itemize}

Sqlite files are stored in \$\{HOME\}/src/c/sqlite/:
\begin{itemize}
\item Sqlite source: (sqlite.fossil)
\item Sqlite forum: (sqliteforum.fossil)
\item Sqlite document collation source (docsrc.fossil)
\item Logic tests for sqlite (sqllogictest.fossil)
\end{itemize}

The code is currently short-routed to only fetch all the fossil files, not to parse parameters.
The switchover to parameters will happen very very shortly.

\begin{verbatim}
#!/bin/bash
# v0.1 iterate through my fossils
# v0.2 Add in some git projects
# v0.3 Added in some more sqlite-related items
# v0.3a TODO: split this up like fossilweb.sh
# v0.4 still working on new fossilstuff function - not live yet
# Really needs to be run from the source directory first
# Cannot get TH3 source without a commercial licence, so can't run tests for docsrc

#########
# Notes #
#########
# fossil has source code, forums and a book.
# sqlite has source code, forums, docsrc and a testing harness

MYHOME="/home/viking/src/c/"

# Obligatory help function
function dohelp() {
    echo "$0: help page"
	echo "$0 fossil-scm [code|forum|book]"
	echo "$0 sqlite [code|forum|docsrc|tests}"
	exit 0
}

function code() {
	cd "${MEPATH}" 
	fossil pull -R ${t%%-scm}.fossil
	cd -
}

function forum() {
	cd "${MEPATH}" 
	fossil pull -R ${t%%-scm}forum.fossil 
	cd -
}

function doc() {
	cd "${MEPATH}" 
	fossil pull -R ${t}.fossil 
	cd -
}

function book() {
	# only true for fossil, not for sqlite
	cd "${MEPATH}" 
	fossil pull -R ${t%%-scm}-book.fossil 
	cd -
}

# New fossilstuff function
function dofossilstuff() {
# (roughly) duplicate what fossilstuff did
# At the moment, only processes one pair of args, then exits.
    pushd "${MYHOME}"
    if [ ${#*} -lt 1 ]; then # do we barf or do all?
	dohelp
    fi
	# We haven't exited yet
	case "${1}" in 
	"fossil-scm"|"fossil")
		MEPATH="fossil-scm"
		case "${2}" in
			"all") code
				forum
				# Leave out book, we can call that manually
				;;
			"code") code ;;
			"forum") forum ;;
			"book") book ;;
			"*") help ;; # everything else gets the boot
		esac
	    ;;
	"sqlite")
		case "${2}" in
			"all") code
				forum
				;;
			"code") code ;;
			"forum") forum ;;
			"*") help ;; # everything else gets the boot
		esac
    ;;
		"docsrc")  # Only the one fossil file here,  not two as in the fossil/sqllite codebases/forums.
			MEPATH="sqlite"
		doc
	;;

	"*") # everything else gets the boot
		help ;;
    esac

}

# Takes path arg
function fossilstuff() {
	pushd "${MYHOME}"
	for t in fossil-scm sqlite; do  # want subshell - saves us a pushd/popd
		case ${t} in "fossil-scm")
			cd ${t}
			fossil pull -R ${t%%-scm}.fossil
			fossil pull -R ${t%%-scm}forum.fossil 
			fossil pull -R ${t%%-scm}-book.fossil 
			cd ..
		;;
		"sqlite") 
			cd ${t}
			fossil pull -R ${t%%-scm}.fossil
			fossil pull -R ${t%%-scm}forum.fossil 
			fossil pull -R docsrc.fossil
			fossil pull -R sqllogictest.fossil
			cd ..
		;;
		esac		
	done
	popd
}


fossilstuff


\end{verbatim}

\subsection*{fossilweb.sh}
\label{sec:org1cfe59b}
This starts up the web servers related to fossil code, forums and the fossil book.
\begin{verbatim}
#!/bin/bash
# v0.0.1 FossilWeb - brings up all fossil servers on 8100/8110/8120
# v0.1.0 Starts up what we choose
# v0.1.2 Removed book from "all" as this very rarely gets updated
# v0.1.3 TODO: Add code to check for already running servers, dump if so

FOSSILHOME="/home/viking/src/c/fossil-scm"

# First the source code
code() {
    fossil server --port 8100 fossil.fossil &
}

# Now the forums
forum() {
    fossil server --port 8110 fossilforum.fossil &
}

# and last, the book files. need ui for this
book() {
    fossil ui --port 8120 fossil-book.fossil &
}

# Everything except book. Seems a bit redundant.
all() {
	code
	sleep 5
	forum
	sleep 5
	# book # doesn't really need this, so we'll call it specifically
}

# Better provide help, can't call it help because of the builtin
dohelp() {
	echo "$0: help screen. Starts fossil server from files on commandline"
	echo "$0 [all|code|forum|book] ..."
	exit 0
}

# Change to correct directory
pushd "${FOSSILHOME}"

if [ ${#*} -lt 1 ]; then # I want it all
	all # sleep is built in between stages
else #iterate, chuck it in if keyword isn't recognised.
	for t in ${*}; do
		case $t in "-h"|"--help") dohelp ;;
			code) code ;;
			forum) forum ;;
			book) book ;;
			*) dohelp ;; # This exits, no matter what the state of other ${*}
		esac
		sleep 5 # Allow each server to start up before anything else happens
	done
fi

# We all done sah.
popd

\end{verbatim}

\subsection*{sqliteweb.sh}
\label{sec:org50c8fb9}
This starts up the web servers related to sqlite code, forums, docsrc and testing code. As yet,
sqlite.org have not released TH3 as free open source code, so I'm unable to completely fulfil
the "docsrc" requirements. TH3 is most definitely commercial, and probably contributes to helping
with their running costs, alongside the encryption and compression source that they can supply.

\begin{verbatim}
#!/bin/bash
# v0.0.1 FossilWeb - brings up all fossil servers on 8100/8110/8120
# v0.1.0 Starts up what we choose
# v0.1.2 Removed book from "all" as this very rarely gets updated
# v0.1.3 TODO: Add code to check for already running servers, dump if so
# v0.1.4 name change about three versions ago to suit sqlite instead of fossil

SQLITEHOME="/home/viking/src/c/sqlite"

# First the source code
code() {
    fossil server --port 8200 sqlite.fossil &
}

# Now the forums
forum() {
    fossil server --port 8210 sqliteforum.fossil &
}

# and the doc source files
docsrc() {
    fossil server --port 8220 docsrc.fossil &
}

# and the SQL Logic Tests
tests() {
	fossil server --port 8230 sqllogictest.fossil &
}

# Everything
all() {
	code
	sleep 5
	forum
	sleep 5
    docsrc
	sleep 5
	tests
	sleep 5
}

# Better provide help, can't call it help because of the builtin
dohelp() {
	echo "$0: help screen. Starts fossil server from files on commandline"
	echo "$0 [all|code|forum|docsrc|tests] ..."
	echo "all: launch everything below, spaced out by five seconds"
	echo "code: sqlite source code"
	echo "forum: sqlite forums - read-only"
	echo "docsrc: source for generating sqlite document tree"
	echo "tests: sql logic test harness"
	exit 0
}

# Change to correct directory
pushd "${SQLITEHOME}"

if [ ${#*} -lt 1 ]; then # I want it all
	all # sleep is built in between stages
else #iterate, chuck it in if keyword isn't recognised.
	for t in ${*}; do
		case $t in "-h"|"--help") dohelp ;;
			"code") code ;;
			"forum") forum ;;
			"docsrc") docsrc ;;
			"tests") tests ;;
			"all") code
				forum
		docsrc
				tests
				;;
			*) dohelp ;; # This exits, no matter what the state of other ${*}
		esac
		sleep 5 # Allow each server to start up before anything else happens
	done
fi

# We all done sah.
popd

\end{verbatim}

\section*{Further directions}
\label{sec:orgc46b9d3}
( or, future thoughts )
These scripts are mostly finished with, there's only the conversion of the fossilsource.sh to
parameters to really be done. Most of the hard work was already done in the sqliteweb.sh and
fossilweb.sh scripts, so the improvements from those scripts are making their way back to
fossilsource.sh now.
\end{document}