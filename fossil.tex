% Created 2023-04-29 Sat 01:01
% Intended LaTeX compiler: pdflatex
\documentclass[11pt]{article}
\usepackage[utf8]{inputenc}
\usepackage[T1]{fontenc}
\usepackage{graphicx}
\usepackage{longtable}
\usepackage{wrapfig}
\usepackage{rotating}
\usepackage[normalem]{ulem}
\usepackage{amsmath}
\usepackage{amssymb}
\usepackage{capt-of}
\usepackage{hyperref}
\author{brickviking}
\date{\textit{<2023-04-29 Sat 01:00>}}
\title{fossil scripts\\\medskip
\large for source code maintenance and forum viewing on localhost}
\hypersetup{
 pdfauthor={brickviking},
 pdftitle={fossil scripts},
 pdfkeywords={},
 pdfsubject={},
 pdfcreator={Emacs 29.0.50 (Org mode 9.5.4)}, 
 pdflang={English}}
\begin{document}

\maketitle
\setcounter{tocdepth}{2}
\tableofcontents



\section*{Introduction}
\label{sec:orgd1ee123}

This is a set of homebaked scripts to fetch fossil-related source code such as the fossil program
or sqlite3 database. It's currently implemented as bash code, with no real intention to rework it to anything
else. Hey, if it works, \ldots{} right?

\section*{Notes}
\label{sec:org1eb5751}

We should probably make sure the fossil directory and the sqlite directories exist before we
try to tangle code into them. Extracting source from here is as easy as `C-c C-v t`.
I have also added org-auto-tangle, so rendering this on platforms without that may cause one or two issues.

Creating a pdf from this document is as easy as—in emacs, anyhow—`C-c C-e l l`, which will export your
file to a .tex file, then you can run pdflatex on this file to produce a pdf. It's not very good looking
at the moment.

\section*{TODO-things}
\label{sec:org8a09195}

\subsection*{Relevancy of certain packages (a.k.a. how often are they updated)}
\label{sec:org28aca34}

Decide whether to keep fsl and fuel in, as they haven't been updated in quite some time (2018 for fuel,
2019 for fsl)

\subsection*{Update fossilsource to do initial fetching}
\label{sec:orgffbe92c}

That's rather going to depend upon what use people have for the respective code. If you don't need it
for study/learning/testing, generally you won't have bothered with fossil anyhow. But if you happen to
be interested in the state of play for the sqlite3 project, then this is how you'd do it. So.

\subsubsection*{Do we add a clone flag to fossilsource?}
\label{sec:org27c4d57}

We either add one or we don't. If I do, then I have to make sure it's reasonably rock-solid. If I don't
then you fetch them like normal (i.e. `fossil clone \url{https://blahdeblah.kom} yourfossilname.fossil`), and make
sure that fossilsource can update them later. At the moment the concept of where I have the fossils is most
definitely baked in, and this would need to change in any event. That'll be a point for "Future Directions" too.

\section*{Changelog}
\label{sec:org1178f9e}

\begin{itemize}
\item 1.0 Initial fossil.sh created, as version 0.0.1, pretty much just fetched fossil-scm source
code.
\begin{itemize}
\item 1.1 expanded to fossilsource.sh, includes forums, added fossil book a while later.
\item 1.2 included fossilweb.sh, initially starts up servers to access fossil source and forums
\item 1.3-1.4 changes
\item 1.5 added sqlite source code fetching and forums to fossilsource.sh
\item 1.6 added sqlite web server to fossilweb.sh
\item 1.7 split off sqliteweb stuff into its own script (sqliteweb.sh)
\end{itemize}
\item 2.0 expanded fossilsource to create functions for fetching individual components.
\begin{itemize}
\item 2.1 added docsrc (sqlite) to fossilsource.sh and sqliteweb.sh
\item 2.2 added sqllogictests to fossilsource.sh and sqliteweb.sh

\item 2.3 Initial cut of a fossil.org containing document, now that these scripts are getting big
enough.
\item 2.4 Added org-auto-tangle, prodded help switch into fossilsource.sh and extra switches into tclweb.sh
\item 2.5 Added libfossil
\item 2.6 Corrected some mistakes
\item 2.7 Added wasm to fossilsource, changed things around a wee bit
\item 2.8 Also added althttpd to sqliteweb, tidied up case switch on \$@
\item 2.9 Added pikchr to fossilsource, putting it into rarely. Tweaked REVISION in header.
\item 2.10 Added fsl to fossilsource, made note of its "rarely" status in fossilsource. 
Also added fuel - Dead. Dodo. Won't compile here. Weirdly enough, I seem to have got it compiled at some stage, though I've no idea how.
\item 2.11 Tidying up fnc, moved fnc from rarely into libfossil section.
\item 2.12 Added a new function to cover cloudsqlite.
\end{itemize}
\end{itemize}
\newpage

\setcounter{tocdepth}{2}
\tableofcontents

\newpage

\section*{Source files}
\label{sec:orgc2ed861}

These files are responsible for updating source fossil files, or starting up the internal
content web servers related to fossil, sqlite, tcl and tk.

\subsection*{fossilsource}
\label{sec:org655a90e}

This fetches all the code that's currently supplied as .fossil files from sqlite.org, including
the sqlite-related files. Additionally, tcl and tk are available too.
As of November 2022, that's the following files stored in \$\{HOME\}/src/c/fossil-scm:
\begin{itemize}
\item fossil-scm source (fossil.fossil)
\item fossil-scm fossil (fossilforum.fossil)
\item fossil-scm book (fossil-book.fossil)
\end{itemize}

Sqlite files are stored in \$\{HOME\}/src/c/sqlite/:
\begin{itemize}
\item Sqlite source: (sqlite.fossil)
\item Sqlite forum: (sqliteforum.fossil)
\item Sqlite document collation source (docsrc.fossil)
\item The althttpd web server (althttpd)
\item Logic tests for sqlite (sqllogictest.fossil)
\item Wasm/JS documents (wasm.fossil)
\end{itemize}

Tcl and Tk files are stored in \$\{HOME\}/src/c/tcl and tk respectively.

libfossil is in \$\{HOME\}/src/c/libfossil, which contains the sourcecode and forum fossils.
The code is currently in the process of being converted to parse parameters.

As of March 2023, several outlier projects were added, such as fsl, pikchr, cvstrac (direct fossil
predecessor), wapp and tips (tcl/tk project). I also added the fuel project, but it hasn't been
updated since 2018 and doesn't want to compile on my system, though I don't know why. As a
result, I'll stick a stanza in, but won't provide a way to update it from fossilsource. It's dead, Jim.

EDIT: I got a copy to compile and run (it at least showed me a GUI), though I can't quite remember how.

In April 2023, fnc was added in, though it had been on my system a while. Poor thing.

Future plans: 
\begin{itemize}
\item include a verbose switch.
\item add other switch processing.
\item (perhaps) add initial cloning.
\end{itemize}

\texttt{======} Beginning of fossilsource \texttt{======}
\begin{verbatim}
#!/bin/bash
# v0.1 iterate through my fossils
# v0.2 Add in some git projects
# v0.3 Added in some more sqlite-related items
# v0.3a TODO: split this up like fossilweb.sh
# v0.4 still working on new fossilstuff function - not live yet
# v0.5 newfossilstuff got the axe. I've got to build a function to
#      cycle through $1
# v0.6 I added tips to Tcl (Tcl Improvement Proposals)
# v0.6a Added extra line between downloads
# v0.7 add libfossil, correct mistakes
# v0.8 create a rarely section for docsrc, book and others
# v0.9 Added sqlite-wasm to the rarely section, also added althttpd
# v0.10 Moved sqlite-wasm back out of rarely, due to activity.
# v0.11 Added pikchr into rarely
# v0.12 Adding params to fossil.
# v0.13 Changed out MYHOME and other vars to FOSHOME etc.
# v0.14 Added sqlite-cloudsqlite, tidied up help
# v0.15 Put in fnc, should have been in there a while. Poor thing.
#########
# Notes #
#########
# Really needs to be run from the source directory first
# fossil has source code, forums, pikchr and a book.
# sqlite has source code, forums, docsrc, althttpd, wasm and a testing harness
# tcl and tk each have source code, and tcl has proposals (tcl-tip)
# Cannot get TH3 source without a commercial licence, so can't run tests for docsrc
# leave libfossil out of the main loop, but call them specifically, just like fossil-book, pikchr, sqlite-wasm and sqlite-testing

FOSHOME="/home/viking/src/c/"
# Obligatory help function
function dohelp() { # needed renaming, as "help" already exists somewhere else
  echo "$0: help page"
  echo "$0 fossil{-scm} [code|forum|book|pikchr]: fetches named section"
  echo "$0 fossil{-scm} all: fetches code, and forum (not book)"
  echo "$0 libfossil{-all}: fetches libfossil (libfossil-code), fnc and libfossil-forum"
  echo "$0 [fsl|fnc]: fetches fsl or fnc"
  echo "$0 sqlite: fetches sqlite fossils (code, forum, docsrc, wasm)"
  echo "$0 sqlite-{code|forum|docsrc|tests|althttpd|wasm}: fetches code, forum, docsrc, althttpd or test scripts"
  echo "$0 tcl-code: fetches code"
  echo "$0 tcl-tip: fetches proposals"
  echo "$0 tk-code: fetches code"
  echo "$0 althttpd: fetches althttpd code"
  echo "$0 wasm: fetches wasm-related code"
  echo "=== Rarely-updated stuff"
  echo "$0 rarely: updates all the stuff not touched by [{fossil|sqlite}-]all"
  echo "$0 Currently, this includes the following fossil and sqlite projects:"
  echo " fossil-book, pikchr, fsl, sqlite-tests, sqlite-althttpd, sqlite-cloud,"
  echo " and cwal. This list will need modifying on occasion"
  echo "======= TODO"
  echo "$0 fsl: fetch from the fsl project"
  echo "$0 {all}: fetches fossil all, sqlite all"
# Need a section in here for pull
  exit 0
}

function fossil-code() {
  fossil ${FOSCMD} ${FOSCMDPARAMS[*]} fossil.fossil
  echo "==== fossil-code...done ==="
}

function fossil-forum() {
  fossil ${FOSCMD} ${FOSCMDPARAMS[*]} fossilforum.fossil 
  echo "==== fossil-forum...done ==="
}

# fossil only. Rarely updated  
function fossil-book() {
  fossil ${FOSCMD} ${FOSCMDPARAMS[*]} fossil-book.fossil 
  echo "==== fossil-book...done ==="
}

# Should go into Rarely
function pikchr() {
  fossil ${FOSCMD} ${FOSCMDPARAMS[*]} pikchr.fossil
  echo "==== pikchr...done ===="
}

function fossil-all() {
  cd fossil-scm # Yes, the directory name has -scm appended
  t=fossil
  ${t}-code
  sleep 5
  ${t}-forum
#  ${t}-book # Shifted to rarely()
  cd ..
}

###### wanderinghorse items
function libfossil-code() {
  fossil ${FOSCMD} ${FOSCMDPARAMS[*]} libfossil.fossil
  echo "==== libfossil...done ==="
}

function libfossil-forum {
  fossil ${FOSCMD} ${FOSCMDPARAMS[*]} libfossil-forum.fossil
  echo "==== libfossil-forum...done ==="
}

function libfossil-all {
  cd libfossil
  libfossil-code
  libfossil-forum
  fnc
  cd -
}

# Scripting engine without a language. Not sure where to put this.
function cwal {
  fossil ${FOSCMD} ${FOSCMDPARAMS[*]} cwal.fossil
}

# cson - old, but seemingly still in use
function cson {
  fossil ${FOSCMD} ${FOSCMDPARAMS[*]} cson.fossil
}

###### end of wanderinghorse stuff

# Shell-based front end to fossil, append to rarely. Hasn't been updated since 2019.
function fsl {
  fossil ${FOSCMD} ${FOSCMDPARAMS[*]} fsl.fossil
}

# ncurses client for fossil
function fnc {
  fossil ${FOSCMD} ${FOSCMDPARAMS[*]} fnc.fossil
}

# This is spectacularly unlikely to update further, but is here for completeness' sake
function fuel {
  cd fossil-scm
  fossil ${FOSCMD} ${FOSCMDPARAMS[*]} fossil-fuel.fossil
  cd -
}

# sqlite-hosted or sqlite-related projects
function sqlite-code() {
  fossil ${FOSCMD} ${FOSCMDPARAMS[*]} sqlite.fossil
  echo "==== sqlite-code...done ==="
}

function sqlite-forum() {
  fossil ${FOSCMD} ${FOSCMDPARAMS[*]} sqliteforum.fossil 
  echo "==== sqlite-forum...done ==="
}

function sqlite-wasm() {
  fossil ${FOSCMD} ${FOSCMDPARAMS[*]} wasm.fossil
  echo "==== sqlite-wasm...done ==="
}

### These fossils are rarely updated. ###
function sqlite-docsrc() {
  fossil ${FOSCMD} ${FOSCMDPARAMS[*]} docsrc.fossil 
  echo "==== sqlite-docsrc...done ==="
}

function sqlite-tests() {
  fossil ${FOSCMD} ${FOSCMDPARAMS[*]} sqllogictest.fossil 
  echo "==== sqlite-tests...done ==="
}

function sqlite-althttpd() {
  fossil ${FOSCMD} ${FOSCMDPARAMS[*]} althttpd.fossil 
  echo "==== sqlite-althttpd...done ==="
}

function sqlite-cloud() {
  fossil ${FOSCMD} ${FOSCMDPARAMS[*]} cloudsqlite.fossil 
  echo "==== sqlite-cloudsqlite...done ==="
}


# Does everything not in "rarely" section
function sqlite-all() {
  t="sqlite"
  cd ${t}
  ${t}-code
  sleep 5
  ${t}-forum
  sleep 5
  ${t}-docsrc
  sleep 5
  ${t}-wasm
#  ${t}-althttpd # shifted to rarely()
#  ${t}-tests # shifted to rarely()
#  ${t}-cloud # shifted to rarely()
  unset $t
  cd .. # gets us back to ${MYHOME}
}

# Updates tcl sourcecode
function tcl-code() {
  fossil ${FOSCMD} ${FOSCMDPARAMS[*]} tcl.fossil 
  echo "==== tcl-code...done ==="
}

# Updates tcl TIP database
function tcl-tips() {
  fossil ${FOSCMD} ${FOSCMDPARAMS[*]} tips.fossil 
  echo "==== tcl-tips...done ==="
}

# Updates tk sourcecode
function tk-code() {
  fossil ${FOSCMD} ${FOSCMDPARAMS[*]} tk.fossil 
  echo "==== tk code...done ==="
}

# Updates both tcl and tk sources
function tcl-all() {
  cd tcl
  tcl-code
  sleep 5
  tcl-tips   # TCL Improvement Proposals, rarely updated
  sleep 5
  cd ../tk
  tk-code
  cd .. # gets us back to ${MYHOME}
}

# these don't get updated very often, if at all
function rarely() { 
# fossil stuff
  cd fossil-scm
  fossil-book
  pikchr
# sqlite stuff
  cd ../sqlite
  sqlite-tests
  sqlite-althttpd
  sqlite-cloud
# wanderinghorse - a.k.a. libfossil
  cd ../libfossil
  cwal
  fsl
  cd ..
}

function all() { # These each have a cd in them
# Note: does not include fsl/libfossil
  fossil-all # code, forum, not book
  sleep 5
  sqlite-all # code, forum, althttpd, not tests
  sleep 5
  tcl-all    # code (tcl/tk), TCL Improvement Proposals (tip)
}
# Takes path arg

pushd "${FOSHOME}"
# Let's add something to FOSCMD to make the operation specific
# the caseloop below processes in order. -u needs to be first if
# we want to sync unversioned artifacts
declare FOSCMDPARAMS
# Was here for getting everything and breadcrumbing where I was
# FOSCMDPARAMS=("-v" "--verily" "-R")
# Strip it back a bit, I think.
FOSCMDPARAMS=("-R")

FOSCMD="pull" # default value unless it gets changed by -u
if [ ${#*} -lt 1 ]; then # I want it all
	all # sleep is built in between stages
else #iterate, chuck it in if keyword isn't recognised.
  for t in ${*}; do
    case "${t}" in 
    "-u"|"--unversioned")
      # sync can't push without user perms on remote, but can still pull unversioned artifacts
      FOSCMDPARAMS=("-u" "-v" "--verily" "-R")
      FOSCMD="sync"
#      dohelp 
    ;;
    "help"|"-h")
      dohelp 
    ;; # exits
      # I should cover this if I want all the ones I don't normally cover
    "rarely") rarely ;; 
    "all") all ;; # Yeah, I know I said it above, but here I specify it.
    "fossil"|"sqlite"|"tcl")
      "${t}"-all
    ;;
# All fossil-related stuff
    "fossil-forum"|"fossil-code"|"fossil-book"|"pikchr")
      cd "fossil-scm"
      "${t}" # calls the function directly. I'll have to see if this works.
      cd ..
    ;;
    "libfossil"|"libfossil-all")
      libfossil-all # should do libfossil-code and libfossil-forum
    ;;
    "cwal")
      cd "libfossil"
      cwal
      cd ..
    ;;
    "cson")
      cd "libfossil"
      cson
      cd ..
    ;;
    "fsl" ) 
      cd "libfossil"
      fsl  # Seems a bit redundant to make this a full function call
      cd -
    ;;
    "fnc" ) 
      cd "libfossil"
      fnc  # Seems a bit redundant to make this a full function call
      cd -
    ;;
    "book")
      cd fossil-scm
      fossil-book 
      cd -
    ;;

# All sqlite-related stuff
"sqlite-code"|"sqlite-forum"|"sqlite-docsrc"|"docsrc"|"sqlite-tests"|"sqlite-althttpd"|"sqlite-wasm") 
      cd sqlite
      "${t}"
      cd -
    ;;
    "althttpd"|"httpd") 
	cd sqlite
	sqlite-althttpd
	cd ..
    ;;
    "docsrc")
	cd sqlite
	sqlite-docsrc
	cd ..
    ;;
    "wasm")
	cd sqlite
	sqlite-wasm
	cd ..
    ;;
    "cloud"|"cloudsqlite"|"sqlite-cloud")
	cd sqlite
	sqlite-cloud
	cd ..
    ;;
# All tcl/tk-related stuff
    "tcl"|"tcl-all")
      tcl-all
    ;;
    "tcl-code"|"tcl-tips")
      cd tcl
      "${t}"
      cd -
    ;;
    "tk")
      cd "${t}"
      "${t}"-code  # Only the code at the moment.
      cd -
    ;;
    "tk-code") 
      cd tk; pwd
      "${t}"
      cd -
    ;;
    esac # end of case ${t}
  done
fi
popd


\end{verbatim}
\texttt{======} End of fossilsource \texttt{======}

\subsection*{fossilweb.sh}
\label{sec:orgccba62f}

This starts up the web servers related to fossil code, forums and the fossil book. It—along with tcl-web/tkweb and sqliteweb have all been condensed into one fossil-driven front end. The only real liability is that I may or may not be able to run multiple tabs each with its own server. I'll have to try that. fossil tends to spawn its subprocesses in a separate tab each time.

\texttt{======} Beginning of fossilweb.sh (superceded) \texttt{======}

\begin{verbatim}
#!/bin/bash
# v0.0.1 FossilWeb - brings up all fossil servers on 8100/8110/8120
# v0.1.0 Starts up what we choose
# v0.1.2 Removed book from "all" as this very rarely gets updated
# v0.1.3 TODO: Add code to check for already running servers, dump if so

# TODO: we need to bring this back up to date. It's been a really long time since July 2022
# Forget it, server mode works far better.
FOSSILHOME="/home/viking/src/c/fossil-scm"

# First the source code
code() {
  echo -ne "Starting fossil code server: "
  fossil server --port 8100 fossil.fossil &
}

# Now the forums
forum() {
  echo -ne "Starting fossil forum server: "
  fossil server --port 8110 fossilforum.fossil &
}

# and last, the book files. need ui for this
book() {
  echo -ne "Starting fossil book server: "
  fossil server --port 8120 fossil-book.fossil &
}

# Everything except book. Seems a bit redundant.
all() {
  code
  sleep 5
  forum
  sleep 5
  # book # doesn't really need this, so we'll call it specifically
}

# Better provide help, can't call it help because of the builtin
dohelp() {
  echo "$0: help screen. Starts fossil server from files on commandline"
  echo "$0 [all|code|forum|book] ..."
  exit 0
}

# Change to correct directory
pushd "${FOSSILHOME}"

if [ ${#*} -lt 1 ]; then # I want it all
  all # sleep is built in between stages
else #iterate, chuck it in if keyword isn't recognised.
  for t in ${*}; do
    case $t in "-h"|"--help") dohelp ;;
      "code") code ;;
      "forum") forum ;;
      "book") book ;;
      "all") all ;; # doesn't include book, call that separately
      *) dohelp ;; # This exits, no matter what the state of other ${*}
		esac
		sleep 5 # Allow each server to start up before anything else happens
	done
fi

# We all done sah.
popd

\end{verbatim}

\subsection*{sqliteweb.sh}
\label{sec:org3f627ec}
-- Now superfluous --
This starts up the web servers related to sqlite code, forums, docsrc, wasm/js and testing code. As yet,
sqlite.org have not released TH3 as free open source code, so I'm unable to completely fulfil
the "docsrc" requirements. TH3 is most definitely commercial, and probably contributes to helping
with their running costs, alongside the encryption and compression source that they can supply.
These have been replaced with a front end on 8100 that points to every fossil on my system.

\begin{verbatim}
#!/bin/bash
# v0.0.1 FossilWeb - brings up all fossil servers on 8100/8110/8120
# v0.0.2 Sqlite fossil servers adjusted to start on 8200/10/20/30
# v0.1.0 Starts up what we choose
# v0.1.2 Removed book from "all" as this very rarely gets updated
# v0.1.3 TODO: Add code to check for already running servers, dump if so
# v0.1.4 name change about three versions ago to suit sqlite instead of fossil
# v0.1.5 Added, then removed TCL Improvement Proposals (TIP), shifted to tclweb.sh
# v0.1.6 Added wasm/js server. This should by rights not be in all,
#        but I'll leave it there for now, as there's recent traffic.
# v0.1.7 TODO: Added althttpd fossil and streamlined the case switch.
SQLITEHOME="/home/viking/src/c/sqlite"

# First the source code
code() {
  echo -ne "Starting SQlite3 code fossil server: "
  fossil server --port 8200 sqlite.fossil &
}

# Now the forums
forum() {
  echo -ne "Starting SQlite3 forum fossil server: "
  fossil server --port 8210 sqliteforum.fossil &
}

# and the doc source files
docsrc() {
  echo -ne "Starting SQlite3 docsrc fossil server: "
  fossil server --port 8220 docsrc.fossil &
}

# and the SQL Logic Tests
tests() {
  echo -ne "Starting SQlite3 test code fossil server: "
  fossil server --port 8230 sqllogictest.fossil &
}

wasm() {
  echo -ne "Starting SQlite3 wasm/JS fossil server: "
  fossil server --port 8240 wasm.fossil &
}

althttpd() {
  echo -ne "Starting SQlite3 althttpd.fossil server: "
  fossil server --port 8250 althttpd.fossil &
}

# Everything
all() {
  code
  sleep 5
  forum
  sleep 5
  docsrc
  sleep 5
  tests
  sleep 5
  wasm
  sleep 5
  althttpd
  sleep 5
}

# Better provide help, can't call it help because of the builtin
dohelp() {
  echo "$0: help screen. Starts fossil server from files on commandline"
  echo "$0 [all|code|forum|docsrc|test|wasm|althttpd] ..."
  echo "all: launch everything below, spaced out by five seconds"
  echo "code: sqlite source code"
  echo "forum: sqlite forums - read-only"
  echo "docsrc: source for generating sqlite document tree"
  echo "tests: sql logic test harness"
  echo "wasm: sqlite3 wasm/js code reference"
  echo "althttpd: sqlite3 althttpd reference"
  exit 0
}

# Change to correct directory
pushd "${SQLITEHOME}"

if [ ${#*} -lt 1 ]; then # I want it all
  all # sleep is built in between stages
else #iterate, chuck it in if keyword isn't recognised.
  for t in ${*}; do
    case $t in "-h"|"--help") dohelp ;;
      "code"|"forum"|"docsrc"|"tests"|"wasm"|"althttpd") "${t}" ;;
      "all") all ;; # Streamlined a bit
      *) dohelp ;; # This exits, no matter what the state of other ${*}
    esac
    sleep 5 # Allow each server to start up before anything else happens
  done
fi

# We all done sah.
popd

\end{verbatim}

\subsection*{tclweb.sh}
\label{sec:orgba787cf}
And the third member of the group, runs servers for the tcl/tk source trees. Once again, replaced
by front end on :8100.

\begin{verbatim}
#!/bin/bash
# v0.0.1 FossilWeb - brings up all fossil servers on 8100/8110/8120
# v0.1.0 Starts up what we choose
# v0.1.2 Removed book from "all" as this very rarely gets updated
# v0.1.3 TODO: Add code to check for already running servers, dump if so
# v0.1.4 name change about three versions ago to suit tcl instead of fossil
# v0.1.5 made notes about starting on ports 8300/10/20

TCLHOME="/home/viking/src/c/tcl"
TKHOME="/home/viking/src/c/tk"

# First the source code
tcl-code() {
  cd "${TCLHOME}"
  echo -ne "Starting Tcl fossil server: "
  fossil server --port 8300 tcl.fossil &
  cd -
}

tcl-tips() {
  cd "${TCLHOME}"
  echo -ne "Starting Tcl Improvement Proposals fossil server: "
  fossil server --port 8310 tips.fossil &
  cd -
}

tk-code() {
  cd "${TKHOME}"
  echo -ne "Starting Tk fossil server: "
  fossil server --port 8320 tk.fossil &
  cd -
}

# Everything
all() {
  tcl-code
  sleep 5
  tcl-tips
  sleep 5
  tk-code
  sleep 5
}

# Better provide help, can't call it help because of the builtin
dohelp() {
	echo "$0: help screen. Starts fossil server for Tcl code from files on commandline"
  echo "$0 [all|tcl-code|tcl-tip|tk-code]"
  echo "all: launch all servers, spaced out by five seconds"
  echo "tcl-code: tcl source code"
  echo "tcl-tips: Tcl Improvement Proposals"
  echo "tk-code: tk source code"
#	echo "forum: tcl forums - read-only"
#	echo "docsrc: source for generating tcl document tree"
#	echo "tests: sql logic test harness"
	exit 0
}

# Change to correct directory
pushd "${TCLHOME}"

if [ ${#*} -lt 1 ]; then # I want it all
	all # sleep is built in between stages
else #iterate, chuck it in if keyword isn't recognised.
  for t in ${*}; do
    case $t in "-h"|"--help") dohelp ;;
      "tcl-code"|"tcl") tcl-code ;;
      "tcl-tips"|"tips") tcl-tips ;;
      "tk-code"|"tk") tk-code ;;
      "all") all ;;
      "*") dohelp ;; # This exits, no matter what the state of other ${*}
    esac
    sleep 5 # Allow each server to start up before anything else happens
  done
fi

# We all done sah.
popd

\end{verbatim}

\subsection*{fossilserve}
\label{sec:org7710993}
This effectively replaces the twelve ports taken up by the individual servers and replaces them with a front listing, which is probably how the fossil creators intend it to be used. 

\begin{verbatim}
#!/bin/bash
# v0.1 Initial release
# we don't exactly need a home, just a straight up fossil invoke
# from the / we can invoke a server (in a new tab) for every fossil we have
#
# TODO: we don't test for already-running fossil. -test is now ancient.

if [[ "$1" == "-test" ]]; then
  fossil-test server --port 8100 / &
else # either there's no $1 or $1 holds something that's not -test
  fossil server --port 8100 / &
fi
\end{verbatim}

\subsection*{fossilgitcreate}
\label{sec:org8b13636}
This little scriptlet will hopefully duplicate a git repo, seeing as there isn't an obvious way
to run fossil git import yet, as it hasn't been fully developed.
\subsubsection*{Problems:}
\label{sec:org078e998}
This doesn't deal at all with any branching. I don't yet know how to do that, but I've no doubt I can
probably coerce both git and fossil to do branch wrangling.
I could possibly write this easier in tcl if I knew how.
\subsubsection*{Source file}
\label{sec:orgd66fd6a}
My concept is as follows:

\begin{itemize}
\item Create a new fossil file in the directory we wish to import from. (This can and should be changed later)
\end{itemize}
\begin{verbatim}
#!/bin/bash
# Creates new fossil from git tree. There'll be a corresponding one for fossilgitadd. Soon.
#
# v0.0.1 Initial cut at problem

# Let's get a repo going. Later we'll ask for the name of this repo ($1 etc)
REPONAME=chooseyournewreponame.fossil
fossil init ${REPONAME} --template ${HOME}/fossils/skeleton.fossil
# Add correct password to setup user, probably ${USER}. We should add a daily-use user later with caps of ay.
fossil user password ${USER} "whateveryoulike" -R ${REPONAME}
# Now we "open" a new dir. Gotta coerce fossil to open in a not-empty directory.
fossil open -R ${REPONAME} --force

# Now we stash the current state, we'll need to come back to here at the end.
git stash create "CurrentState"
\end{verbatim}
\begin{itemize}
\item Add user if not set from template
\item Run git log, look for very first entry. check that out.
\item fossil add all the files that turn up, excluding the fossil file itself, the .git hierarchy and the .fslckout, obviously.
\item loop through all the other git leaves
\begin{itemize}
\item adding them to fossil one by one
\item with the commit message duplicating the one held by git.
\end{itemize}
\item close out the fossil, as the last git commit should now be at HEAD
\end{itemize}

\section*{Further directions}
\label{sec:org3dba31d}
( or, future thoughts )
These scripts are mostly finished with, and the conversion of the fossilsource.sh to
parsing parameters is now done. Most of the hard work was already done in the sqliteweb.sh and
fossilweb.sh scripts, so the improvements from those scripts have made their way back to
fossilsource.
fossilweb, sqliteweb, and tclweb have all been pretty much superseded by fossilserve, as it just shows a directory of all my fossils and lets me choose, instead of running up an individual server on a separate port for the thing I want.
\end{document}